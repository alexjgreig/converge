\let\textcircled=\pgftextcircled
\chapter{Project Work Evidence}
In conjunction with my commits on github at the link: \href{https://github.com/alexjgreig/converge}{\textcolor{blue}{Converge Github}}, I have created a logbook documentation to highlight major milestones in the process of creating the software application and my thought process behind the decisions I made along the way. \\

\section{10/4/22}
I began the project with the idea of Unison. A software solution that sophisticatedly shares computational load across website visitors to achieve a common goal. Utilising WebAssembly and the Rust Programming Language, Unison efficiently shares computational load and would demonstrate its capability through community cryptocurrency mining. This was the beginning point for this project and was a major milestone as it marked the start of this ever evolving and shifting project. \\
\section{27/4/22}
At this point in time, I had researched quite a bit into how WebAssembly functions and how one might achieve a sharing of computational load. I was going to create a front-end interface to access a WebAssembly(WASM) binary that would solve computational problems for point of work cryptocurrency mining, the solutions and which cryptographic nonces to check would be transmitted between a user and a central server and thus would be, “sharing computational load.” Although this idea interested me and the technology was fascinating, I wanted to move towards a more financial related software project as I did not see a lot of value in what I was currently creating. This is where I moved onto Converge. This project idea was to sophisticatedly display the global economy using data-driven machine learning. It would provide a macroeconomic view of the economy using sentiment analysis of market news and utilise statistical methods to devise an arbitrary value of economic potential for each country. This was more financial than my current project and I could see how this would be used for financial institutions to see a macro view of each country’s economies. I began researching how I would retrieve market information, undergo dimensionality reduction, and then produce an arbitrary value for each economy. I then researched the different web frameworks, such as ReactJS, to build a visually appealing front end to display all this data. \\
\section{10/5/22}
At this point in time, I had made some progress on my application. Finding how to retrieve data from the market and news sources using web-scraping and certain API’s such as NASDAQ’s Data Link. Further I had researched how to intertwine WebAssembly and Javascript to be able to develop a system that figures out the arbitrary value of the economy and then display it by communicating to JavaScript. Although this was an interesting project with lots of potential and full of interesting technology such as machine learning, statistical models, WebAssembly and frontend development it deviated from the area that I wished to develop a project for. The project was a lot more economical in nature and converge, “diverged,” from the business / financial industry that I wanted to develop for. At this point in time, I was unsure of whether to just continue and finish the project and be not truly satisfied with what I had created. This left me in a state of ambivalence as I did not have another idea to shift to and there was not much time left to be switching ideas.  \\

This is where I bought a car, and unfortunately (or fortunately looking back in hindsight as without it this project may never have been undertaken) I had to pay a stamp duty to change the registration or ownership of the car. This gave me the idea to create a blockchain in which people could transfer assets that were tokenised with the efficiency and security inherent with blockchain technology. Further this would allow buyers to see the full history of the car and its previous owners, a tamperproof history of each car around the world. This project idea included sophisticated, bleeding edge technology, however, it was not for the financial businesses industry that I wanted to create software for. This is when the idea of Converge was created. A decentralised blockchain for sophisticated asset management and cross-business transactions. Utilising an intelligent consensus mechanism and concepts such as tokenisation, converge will efficiently create a complete, transparent, tamperproof history and facilitation of the information flows, inventory flows, and financial flows in transactions. I decided to keep the name Converge as it represented how the businesses would come together, “Converge” into one business blockchain where they could transact with each other globally, with minimal costs, blazing fast execution and with the utmost security. At this point I began the long process of researching how blockchain technology worked, the complex and intricate nature of how a business block was created, how asset transactions worked, how assets could be tokenised, the legal proceedings and regulations, the consensus mechanisms, the intricate runtime and networking mechanisms and finally how I would connect it all together into a replicable node that formed a global network that could be accessed, from anywhere, on visually appealing, easy-to-use front-end. \\
\section{11/5/22}
The next day after my idea creation I began to try and build a foundational structure for the blockchain which would take multiple parameters such as block\_id, hash…etc. At this point I realised the sheer size of the software application that I was trying to create. I was competing with the biggest firms in the world who have contributed billions of dollars to research: McKinsey, PwC, Deloitte, EY, Accenture, KPMG… I was creating a global network that would connect the top business’s together on one business blockchain to facilitate transactions, a network that would become a complete, transparent, tamperproof history and facilitation of the information flows, inventory flows, and financial flows across the globe. Trillions of dollars (in concept) could be flowing through the network at a single point in time. The enormity of the project sunk in and I decided that I wouldn’t be able to start from scratch, create all the intricate runtime details, all the peer to peer protocol implementation and networking logic, I needed to work with a framework to actualise my idea. \\
\section{23/5/22}
This was a major milestone in my project as I found the framework that I would use for my project. Substrate is a blockchain framework that enables developers to build customised blockchains while providing the backbone utility to help it run. It provides abstractions over complex processes such as peer to peer networking, remote procedure calls and WASM Runtime modules. This enabled me to focus on what I wanted to create rather than all the complex, intricate modules that was needed to get a basic, foundational model running. \\
\section{8/6/22}
At this point in time, I had finished all of my research surrounding how business blockchains operate and how the technology interconnects with each other to create a self-sustaining system. I had also finalised all the intricate details surrounding the product I was creating, including how the tokenisation process would work, how all the business’s would connect with each other and how they would share assets. \\
\section{10/6/22}
This was a major milestone in the project as I added the ability for the blockchain to utilise non-fungible tokens, which will be used to transfer assets between businesses. The assets specifically that businesses will use this token for is things like contracts, consultancy hours, a specific building, or anything that is unique in nature. I also updated the front-end to display these operations and updated the chain specification to be further targeted towards the business blockchain. \\
\section{12/6/22}
This was another milestone in the development of Converge as I added the ability for businesses to mint and issue fungible assets that represented their physical assets. This was then added to the front end of the blockchain so that businesses could be added. I also added the ability for businesses to be added to the blockchain while it is running through the utilisation of extrinsics. Instead of having to have all the business public and private keys before the blockchain was started, which was infeasible, if business’s want to join the blockchain the network does not need to be shut down but they can be added and confirmed by well-known nodes in the network. 
\section{19/6/22}
This was another major milestone as I had fixed all errors with the node software and a minimal viable product had been created, although there was some features that I wanted to add to the software solution. At this point I needed to ensure that the front end worked in tandem with the node software and after this was done I could implement more complex smart contracts and possibly zero knowledge proofs if time allowed.
